%*******************************************************************************
%*********************************** Third Chapter *****************************
%*******************************************************************************
\chapter{Chapter Three}
\chaptermark{Chapter 3}
\label{Chap3}

\section{Introduction}

\blindtext[5]

\section{Some mathematical section}

\begin{theorem}
The Pythagoras theorem can be stated as follows,
\begin{equation}
a^{2} = b^2 + c^{2}, 
\end{equation}
where $a$ represents the hypotenuse of a rectangle triangle and $b$ and $b$ represent the other sides.
\end{theorem}

\begin{proof}
The proof was given by Pythagoras and is available in the following paper \citep{Veljan2000}.
\end{proof}

Pythagoras theorem can be represented with a nice Tikz picture as presented in \ref{fig1} and available at this url: \\
\url{http://www.texample.net/tikz/examples/pythagoras-triangle/}.

\begin{figure}
\centering
\newcommand{\pythagwidth}{3cm}
\newcommand{\pythagheight}{2cm}
\begin{tikzpicture}
  \coordinate [label={below right:$A$}] (A) at (0, 0);
  \coordinate [label={above right:$B$}] (B) at (0, \pythagheight);
  \coordinate [label={below left:$C$}] (C) at (-\pythagwidth, 0);

  \coordinate (D1) at (-\pythagheight, \pythagheight + \pythagwidth);
  \coordinate (D2) at (-\pythagheight - \pythagwidth, \pythagwidth);

  \draw [very thick] (A) -- (C) -- (B) -- (A);

  \newcommand{\ranglesize}{0.3cm}
  \draw (A) -- ++ (0, \ranglesize) -- ++ (-\ranglesize, 0) -- ++ (0, -\ranglesize);

  \draw [dashed] (A) -- node [below] {$b$} ++ (-\pythagwidth, 0)
            -- node [right] {$b$} ++ (0, -\pythagwidth)
            -- node [above] {$b$} ++ (\pythagwidth, 0)
            -- node [left]  {$b$} ++ (0, \pythagwidth);

  \draw [dashed] (A) -- node [right] {$c$} ++ (0, \pythagheight)
            -- node [below] {$c$} ++ (\pythagheight, 0)
            -- node [left]  {$c$} ++ (0, -\pythagheight)
            -- node [above] {$c$} ++ (-\pythagheight, 0);

  \draw [dashed] (C) -- node [above left]  {$a$} (B)
                     -- node [below left]  {$a$} (D1)
                     -- node [below right] {$a$} (D2)
                     -- node [above right] {$a$} (C);
\end{tikzpicture}
\caption{Representation of the Pythagoras theorem.}
\label{fig1}
\end{figure}